%% Documentation des KOMA-Skriptes
%% http://www.rrzn.uni-hannover.de/fileadmin/kurse/material/latex/scrguide.pdf

	\newcommand{\version}{V 0.6.3 - 6.11.2013 - Derdau/Hecht }

%0.2  Stichwortverzeichnis geändert
%     vor \printindex noch einzufügen:
%    \renewcommand{\indexname}{Stichwortverzeichnis}
%	
%	  Hyperref zugefügt

%0.3  Kurzformen für N,Z,Q,R,C,F 
%0.4  Maple-Programmcode hinzugefügt
%     Zahlenstrahlrealisierung 
%     defienum hinzugefügt

 %%%%%%%%%%%%%%%%%%%%%%%%%%%%%%%%%%%%%%%%%%%%%%%%%%%%%%%%%%%%%%%%%%%%%%%%%%
 % Einbinden weiterer Pakete für Deutsch
  	\usepackage[ngerman]{babel}               % entsprechend fuer die neue Rechtschreibung
  	\usepackage[T1]{fontenc}                     % Sonderzeichen
	\usepackage[utf8]{inputenc} 		%falls Sie Umlaute in den Quellen verwenden wollen
	\usepackage{lmodern}
	%\usepackage[tracking=true]{microtype}
	%\usepackage{txfonts}
	%\usepackage{pxfonts}
	%\usepackage{dsfont}
	
% IFTHEN wird für eigene Anpassungen an Figure und table benötigt
	\usepackage{ifthen}
	%\allowdisplaybreaks

%%% PAGE DIMENSIONS
	\usepackage{geometry} % to change the page dimensions
 	\geometry{a4paper} % or letterpaper (US) or a5paper or....
\geometry{left=2.625cm,right=1.95cm,top=1.5cm,bottom=3.6cm,includehead}
 	 %\geometry{margin=0.8in} % for example, change the margins to 2 inches all round 
	 % \geometry{landscape} % set up the page for landscape 
	%   read geometry.pdf for detailed page layout information

          % Activate to begin paragraphs with an empty line rather than an indent
          % \usepackage[parfill]{parskip}
	% Alternativ 1.Zeile eines Absatzes nicht eingerückt
 	\parindent0cm 

           %Zeilenabstände: 
           \linespread {1.25}\selectfont %1.25 da er von Haus aus 1.2 ist und 1,25 * 1,2 = 1,5 ist

           %Bildunterschriften: 
           \addtokomafont{caption}{\small} %KOMA Syntax

\setlength{\headheight}{2\baselineskip}



%%% Nuetzliche Pakete
 	\usepackage{booktabs} % for much better looking tables
	\usepackage{array} % for better arrays (eg matrices) in maths
 	\usepackage{paralist} % very flexible & customisable lists (eg. enumerate/itemize, etc.)
	\usepackage{verbatim} % adds environment for commenting out blocks of text & for better verbatim
 	\usepackage{subfig} % make it possible to include more than one captioned figure/table in a single float
    %\usepackage[bookmarks,bookmarksnumbered=TRUE]{hyperref} % Erstellt Hyperreferenzen im PDF-Dokument
    %\usepackage[bookmarks,bookmarksnumbered=TRUE,hidelinks]{hyperref}
    % Versteckt den roten Rahmen
 	% These packages are all incorporated in the memoir class to one degree or another...
%\usepackage{showframe}

%%% AMS-Pakete
	% enthaelt nuetzliche Makros fuer Mathematik
	\usepackage{amsmath}
	% fuer Saetze, Definitionen, Beweise, etc.
	\usepackage{amssymb}
	\usepackage{amsthm}
	\usepackage{xfrac}
	
	
	
%Timo
%\usepackage[mathrmOrig]{sfmath} % Option auf Keine Kursive Mathe Schrift



%%% Deklaration eigener Satz-/Definitions-/Beweisumgebungen mit amsthm
% ftp://ftp.ams.org/pub/tex/doc/amscls/amsthdoc.pdf
 %http://matheplanet.com/default3.html?call=viewtopic.php?topic=139377&ref=http%3A%2F%2Fwww.google.com%2Furl%3Fsa%3Dt%26rct%3Dj%26q%3D%2522Deklaration%2Beigener%2BSatz-%252FDefinitions-%252FBeweisumgebungen%2Bmit%2Bamsthm%2522%26source%3Dweb%26cd%3D7%26ved%3D0CEgQFjAG
  	\theoremstyle{definition}
  	
  	%21. November
\newboolean{schatten}
\setboolean{schatten}{true} %Schaltet den Schatten ein/aus
\usepackage{shadethm}
%\if \boolean{boolvar}{

\usepackage{mdframed}


\ifthenelse{\boolean{schatten}}{
	\newshadetheorem{satz}{Satz}[section]
  	\newshadetheorem{theorem}[satz]{Theorem}
  	\newshadetheorem{hilfs}[satz]{Hilfssatz}
  	\newshadetheorem{lemma}[satz]{Lemma}
  	\newshadetheorem{korollar}[satz]{Korollar}
  	\newshadetheorem{proposition}[satz]{Proposition}
  	\newmdtheoremenv[backgroundcolor =black!25!white, linecolor = white, innerleftmargin=0.25cm,innerrightmargin=0.25cm,innertopmargin = 0.05cm,innerbottommargin = 0.25cm,everyline=true]{defi}[satz]{Definition}
  	\newshadetheorem{folg}[satz]{Folgerung}
  	\newshadetheorem{erin}[satz]{Erinnerung}
  	\newshadetheorem{verein}[satz]{Vereinbarung}
}{
	\newtheorem{satz}{Satz}[section]
  	\newtheorem{theorem}[satz]{Theorem}
  	\newtheorem{hilfs}[satz]{Hilfssatz}
  	\newtheorem{lemma}[satz]{Lemma}
  	\newtheorem{korollar}[satz]{Korollar}
  	\newtheorem{proposition}[satz]{Proposition}
  	\newmdtheoremenv[linecolor = white,everyline=true]{defi}[satz]{Definition}
  	\newtheorem{folg}[satz]{Folgerung}
  	\newtheorem{erin}[satz]{Erinnerung}
  	\newtheorem{verein}[satz]{Vereinbarung}
}


  	\newtheorem{bem}[satz]{Bemerkung}
  	\newtheorem{bsp}[satz]{Beispiel}
  	\newtheorem*{bsp_on}{Beispiel} % Ohne Nummerierung
  	\newtheorem*{bem_on}{Bemerkung} % Ohne Nummerierung
 	\newtheorem*{bsp_eng}{Example} % Ohne Nummerierung
  	\newtheorem{satz+def}[satz]{Satz und Definition}
  	\newtheorem{def+lem}[satz]{Definition und Lemma}
  	\newtheorem{war}[satz]{Warnung}
  	%\newtheorem{algo}[satz]{Algorithmus}
	%\newmdtheoremenv[frametitle={blabla},frametitlerule=true,frametitlebackgroundcolor=gray!20,linecolor=black,linewidth=2pt]{algo}[satz]{Algorithmus}
\mdtheorem[frametitlerule=true,frametitlebackgroundcolor=blue!30,linecolor=black,linewidth=2pt]{algo}[satz]{Algorithmus}
  	\newtheorem{bembsp}[satz]{Bemerkung und Beispiel}
  	\newtheorem{bez}[satz]{Bezeichnung}



	\newenvironment{loesung}{\begin{proof}[Lösung]}{\end{proof}}

	\newenvironment{aufgabe}{\vskip
	0.5cm\begin{minipage}[l]{14.7cm}\begin{satz}[Aufgabe]\vskip
	0.1cm}{\end{satz}\end{minipage}}

  	\newenvironment{beweis}%
    		{\begin{proof}[Beweis]}
    		{\end{proof}}
  	\newtheorem{beispiel}[satz]{Beispiel}
    \newenvironment{KommentarBB}%
    		{\begin{proof}[\colorbox{red}{Hinweis:}]}
    		{\end{proof}}
    \newenvironment{KommentarAH}%
    		{\begin{proof}[\colorbox{green}{Hinweis:}]}
    		{\end{proof}}
 
 	% Um einen Doppelpunkt oder einen Zeilenumbruch hinter der Nummer der Definition zu implementieren
 	 \makeatletter
  		\g@addto@macro\thm@space@setup{\thm@headpunct{:\\}}
  	\makeatother

    \newenvironment{defienum}[1][1.]{~\vspace*{-\baselineskip}\begin{enumerate}[#1]}{\end{enumerate}}

%%% EPS-Grafiken einbinden
	\usepackage{graphicx}
	\usepackage{epstopdf}

\usepackage {picins}

%%% Eigene Umgebung für Bilder, die nicht gleiten sollen
	\newenvironment{myfigure}{\begin{minipage}{\textwidth}\vspace*{6pt}\normalfont}{\end{minipage}\vspace*{10pt}}

	\newcommand{\mycaption}[1]{\refstepcounter{figure}%
	  \setbox0=\hbox{\small \textbf{ Abb. \thefigure} #1}%
	  \ifthenelse{\wd0 > \linewidth}%
	  {\begin{minipage}{\textwidth}\vspace*{10pt}\small \textbf{Abb. \thefigure.} #1\end{minipage}}%
	  {\begin{center}\small \textbf{Abb. \thefigure.} #1\end{center}}
	\addcontentsline{lof}{figure}{\numberline{\thefigure}{\ignorespaces #1} } }


%%% Eigene Umgebung für Tabellen, die nicht gleiten sollen
	\newenvironment{mytable}{\begin{minipage}{\textwidth}\vspace*{6pt}\normalfont}{\end{minipage}\vspace*{10pt}}

	\newcommand{\mytablecaption}[1]{\refstepcounter{table}%
	 	\setbox0=\hbox{\small \textbf{Table \thetable:} #1}%
  		\ifthenelse{\wd0 > \linewidth}%
  		{\begin{minipage}{\textwidth}\small \textbf{Table \thetable.} #1\end{minipage}}%
  		{\begin{center}\small \textbf{Table \thetable.} #1\end{center}}
	\addcontentsline{lot}{table}{\numberline{\thetable}{\ignorespaces #1} } }

%%% Englische Beschriftungen

\DeclareNewTOC[
  type=engtable,                         % Name der Umgebung
  types=engtables,                       % Erweiterung (\listofschemes)
  float,                               % soll gleiten
  floatpos=tbp,                        % voreingestellte Gleitparameter
  name=Table,                         % Name in Überschriften
  %listname={Verzeichnis der Schemata}, % Listenname
   counterwithin=section
]{los1}

\DeclareNewTOC[
  type=engfigure,                         % Name der Umgebung
  types=engfigures,                       % Erweiterung (\listofschemes)
  float,                               % soll gleiten
  floatpos=tbp,                        % voreingestellte Gleitparameter
  name=Figure,                         % Name in Überschriften
  %listname={Verzeichnis der Schemata}, % Listenname
   counterwithin=section
]{los2}

% Einige Kurzformen 
\def\N {{\mathbb N}}
\def\Z {{\mathbb Z}}
\def\Q {{\mathbb Q}}
\def\R {{\mathbb R}}
\def\C {{\mathbb C}}
\def\F {{\mathbb F}}
\def\K {{\mathbb K}}
\def\dx {{\, \mathrm{d}x}}
\def\dt {{\, \mathrm{d}t}}
\def\cond {{\mathrm{cond}}}
\def\eps {{\epsilon_M}}

%Paket für Graphen
\usepackage{tikz}
\usetikzlibrary{decorations.pathreplacing}
%Symbol für Widerspruch
\usepackage{stmaryrd}



% Nummerierung der Gleichungen kapitel.nummer
%\def\theequation{\thesection.\arabic{equation}}
	\def\theequation{\arabic{equation}}
	\numberwithin{equation}{section}

% Nummerierung der Abbildungen kapitel.nummer
	%\def\thefigure{\thesection.\arabic{figure}}
	\def\thefigure{\arabic{figure}}
	\numberwithin{figure}{section}

% Nummerierung der Tabellen kapitel.section.nummer
	%\def\thetable{\thesection.\arabic{table}}
	\def\thetable{\arabic{table}}
	\numberwithin{table}{section}

%%% HEADERS & FOOTERS
	\usepackage{scrpage2} % This should be set AFTER setting up the page geometry
	%\pagestyle{plain} % options:  plain , scrheadings, 
	\pagestyle{scrheadings}
          %Kapitelname in Kopfzeile [linke Seite]{rechte Seite}
          \automark[chapter]{chapter}
	%Kopfzeile linke Seite
	\lehead{\headmark} \cehead{}\rehead{}

	%Kopfzeile rechte Seite
	\lohead{}\cohead{}\rohead{\headmark}

	%Fußzeile linke Seite
	\lefoot[\thepage]{\thepage}\cefoot{}\refoot{}
	%Fußzeile rechte Seite
          \lofoot{}\cofoot{}\rofoot[\thepage]{\thepage}
%\renewcommand{\chapterpagestyle}{scrheadings}
%%% Einrichten eines Index
	\usepackage{makeidx}
	%\makeindex
	\newcommand{\rdi}[1]{{\bfseries #1}\index{#1}}
    \newcommand{\rdii}[2]{{\bfseries #2}\index{#1!#2}}

 %% Aufruf von makeindex mit dem Parameter -s mkidx.ist


% Umbenennung der Überschrift für die abstract-Umgebenung


%%% SECTION TITLE APPEARANCE
	\usepackage{sectsty}
	\allsectionsfont{\sffamily\mdseries\upshape} % (See the fntguide.pdf for font help)

%%% ToC (table of contents) APPEARANCE
	%\usepackage[nottoc,notlof,notlot]{tocbibind} % Put the bibliography in the ToC
	\usepackage[titles,subfigure]{tocloft} % Alter the style of the Table of Contents
	\renewcommand{\cftsecfont}{\rmfamily\mdseries\upshape}
	\renewcommand{\cftsecpagefont}{\rmfamily\mdseries\upshape} % No bold!

%%% Überschriften anpassen
%\usepackage{titlesec} \titleformat{\chapter}{\bf\Huge\center}{\thechapter\quad}{0em}{}
%\usepackage{titlesec} \titleformat{\section}{\bf\Huge}{\thesection\quad}{0em}{}
\renewcommand*{\chapterformat}{}
\renewcommand*{\chaptermarkformat}{}
\let\originaladdchaptertocentry\addchaptertocentry
\renewcommand*{\addchaptertocentry}[2]{%
\originaladdchaptertocentry{}{#2}%
}
\renewcommand*{\thesection}{\arabic{section}}

%%% Chapterüberschrift zentrieren
%\renewcommand*{\raggedsection}{\centering}

%%% Section ohne Punkte
\usepackage{titletoc} % Inhaltsverzeichnis anpassen
\titlecontents{section}[3.7em]{}{\contentslabel{2.2em}}{}{\titlerule*[0.3pc]{ }\contentspage}

%%% Titel des Dokumentes

	\title{Brief Article}
	\author{The Author\thanks{FH Bielefeld - Angewandte Mathematik}}
	%\date{} % Activate to display a given date or no date (if empty),
                % otherwise the current date is printed 

%%% TESTBEREICH
%% Version 1

\usepackage{xcolor}
%\usepackage{amsmath}
\definecolor{MapleRed}{rgb}{1,0.0,0.0}
\usepackage[]{listings}
\lstdefinelanguage{maple}
   {morekeywords={true, false, try, catch, return, break, error,%
                  module, export, local, option, in, use,add,seq,sum%
                  and, or, not, xor, xnor,if, then, elif, else, fi,%
                  while, for, from, by, to, do, od,proc, nargs, %
                  local, global, end, NULL,with,restart}}
\lstnewenvironment{lstmapleinput}
     { \renewcommand*\thelstnumber{$\boldsymbol{>}$\hfill}%
       \lstset{language=maple,basicstyle=\color{MapleRed}%
\fontfamily{phv}\selectfont,numbers=left,xleftmargin=3em}%
     }{}


%% Version 2
%Siehe:
  %
   % http://www.mapleprimes.com/questions/37934-Including-Maple-Inputoutput-In-LaTeX
\definecolor{MapleRed}{rgb}{1,0,0}
\definecolor{MapleBlue}{rgb}{0,0,1}
\definecolor{MaplePink}{rgb}{1,0,1} 
\definecolor{MatlabString}{rgb}{1,0,1} 

\def\MapleInput#1{\noindent{{\small $>$ {\tt \color{MapleRed}{#1} }}}}
\def\MapleOutput#1{{\begin{center}\begin{math} \color{MapleBlue}{#1}\end{math} \end{center}}}
\def\MapleWarning#1{\noindent{{\small {\tt \color{MaplePink}{#1} }}}}

\def\MatlabInput#1{\noindent{{{\tt {#1} }}}}


%% Definition Zahlenstrahl
% siehe http://gruppen.niuz.biz/zahlenstrahl-t24578.html
\usepackage{tikz}

%Algorithmen anzeigen
\usepackage{alltt}

%Aufzählung
\usepackage{enumitem}

\usepackage{color}

%\newcolumntype{L}[1]{>{\raggedright\arraybackslash}p{#1}} % linksbündig mit
% Breitenangabe
\newcolumntype{C}[1]{>{\centering\arraybackslash}p{#1}} % zentriert mit Breitenangabe
\newcolumntype{R}[1]{>{\raggedleft\arraybackslash}p{#1}} % rechtsbündig mit Breitenangabe
 
%\newcommand{\il}{\int\limits} %Integral mit Grenzen
\newcolumntype{L}{>{\labelitemi~~}l<{}}

%Definitionen Bachmann
\newcommand{\Int}{\ds\int} 
\newcommand{\Sum}{\ds\sum} 
\newcommand{\Lim}{\ds\lim}
\newcommand{\open}{\stackrel{\circ}}
\renewcommand{\mit}{\;\big|\;}

\newcommand{\ds}{\displaystyle}
%\newcommand{\bx}{\hfill$\rule{2mm}{2mm}$}
\newcommand{\pa}{\partial}
\newcommand{\la}{\lambda}
\newcommand{\ula}{\ul{\lambda}}
\newcommand{\halb}{\frac{1}{2}}
\newcommand{\ol}{\overline}
\newcommand{\ul}{\underline}
\newcommand{\0}{\ul{0}}
\renewcommand{\a}{\ul{a}}
\renewcommand{\b}{\ul{b}}
%\renewcommand{\c}{\ul{c}}
\renewcommand{\d}{\ul{d}}
\newcommand{\s}{\ul{s}}
\newcommand{\x}{\ul{x}}
\newcommand{\y}{\ul{y}}
\newcommand{\qq}{\ul{q}}
%\newcommand{\xAx}{\x^TA\x}
%\newcommand{\xAix}{\x^TA^{-1}\x}
%\newcommand{\xx}{\x^T\x}
%\newcommand{\dxdg}{\Delta \x^T \Delta g}
%\newcommand{\dxhdg}{\Delta \x - H \Delta g}
%\newcommand{\grdqk}{\nabla q(\x_k)}
%\newcommand{\qdef}{\halb\xAx - \b^T\x}
\newcommand{\sgn}{\text{sgn}}

%15. Oktober 2013
%Fußnoten sind am Ende der Seite
%Fußnote ohne Strich
%\let\footnoterule\relax
%kein Strecken der Seite
\raggedbottom
%Minipage wird mit arabischen Ziffern nummeriert
\renewcommand{\thempfootnote}{\arabic{mpfootnote} }
%\usepackage[perpage]{footmisc} %Fußnoten zählet auf jeder Seite Neustarten.
\usepackage{footnpag}
%1. November 2013
%Fuer Gauß-Elimination
\usepackage{gauss}

\makeatletter
\renewcommand*\env@matrix[1][*\c@MaxMatrixCols c]{%
\hskip -\arraycolsep
\let\@ifnextchar\new@ifnextchar
\array{#1}}
\makeatother 

\newmatrix{.}{)}{nurrechts}
\newmatrix{.}{.}{ohne}

%6- November
%fuer LR-Zerlegung
\usepackage{blkarray}
%PDF-Seiten einfügen
\usepackage{pdfpages} 

%22. November
%Hintergrundbild
\usepackage{eso-pic}
\newcommand\BackgroundPic{%
\put(0,0){%
\parbox[b][\paperheight]{\paperwidth}{%
\vfill
\centering
\includegraphics[width=\paperwidth,height=\paperheight,%
keepaspectratio]{images/titel_gruen.jpg}%
\vfill}}}
%variabler Zeilenabstand
\usepackage{setspace}

%11. Dezember
\newcommand{\indexf}[1]{\textbf{#1}\index{#1}}
%\renewcommand{\indexf}[2]{\textbf{#1} \index{#2}}





% Thyra
%\makeatletter
%\renewcommand*\bib@heading{%
%  \section*{Literaturverzeichnis}%
%  \@mkboth{\refname}{\refname}}
%\makeatother


\usepackage{bibgerm} % deutsches Literaturverzeichnis
\bibliographystyle{geralpha} % Stil des Literaturverzeichnis
\usepackage[square,sort,comma,numbers]{natbib}

\usepackage[german,refpage,intoc]{nomencl}

% Mehrere Literaturverzeichnisse
\usepackage{multibib}
\usepackage{etoolbox}

\newcites{HJKmlpBIB}{Literaturverzeichnis}
\newcites{HJKllpBIB}{Literaturverzeichnis}
\newcites{TLpppBIB}{Literaturverzeichnis}
\newcites{TLmeaBIB}{Literaturverzeichnis}

%\newcites{FBKbmoBIB}{Biometric Methods for Power Plant Optimization}
\newcites{FBKbmoBIB}{Bibliography}
\newcites{SPopbBIB}{Literaturverzeichnis}
\newcites{CClrbBIB}{Literaturverzeichnis}
\newcites{ZPsoBIB}{Bibliography}


\BeforeBeginEnvironment{thebibliography}{%
  \let\origchapter\chapter% save the original definition of \section
  \let\chapter\section%  make \section behave like \subsection
}


\AfterEndEnvironment{thebibliography}{%
  \let\chapter\origchapter% restore the original definition of \section
}


% Damit auch Abschnitt x.x.x.x nummeriert wird
\setcounter{secnumdepth}{5}
\setcounter{tocdepth}{5}

% Um die geschwungenen Buchstaben u,k und w darzustellen
\newcommand*\lateinausg{\fontfamily{wela}\selectfont}  

% Zum hervorheben von Textpassagen 
\usepackage{ulem}

% Kleines Summenzeichen zur Darstellung im tiefgestellten Modus
\def\Sum{\scalebox{0.5}{$\sum$}}
\newcommand{\las}[1]{\large{\textsl{\lateinausg{#1}}}}
\newcommand{\ib}[1]{\textbf{\textit{#1}}}

\usepackage{nccmath} % für linksbündige Formeln  
\usepackage{caption} % zum darstellen der unterschriften der grafiken

% scaliertes Array und Matrizen
\newcommand{\colvec}[2][.8]{%
  \scalebox{#1}{$\begin{array}{@{}c@{}}#2\end{array}$}}%
  
\newcommand{\sbmatrix}[2][.8]{%
  \scalebox{#1}{$\begin{bmatrix}{}#2\end{bmatrix}$}}%
  
\newcommand{\spmatrix}[2][.8]{%
  \scalebox{#1}{$\begin{pmatrix}{}#2\end{pmatrix}$}}%
  
% Kurzform
\def\V {{\mathbb V}}
\def\B {{\mathbb B}}
\def\M {{\mathbb M}}

% Um bei Tabellen eine Doppelline zu erzeugen
\usepackage{hhline,float} 

% Farben um die Tabellen zu hinterlegen
\colorlet{mytabgrey1}{black!10}
\colorlet{mytabgrey2}{black!20}

% Um Zeilen, Spalten und Zellen innerhalb von Tabellen einzufärben.
\usepackage{colortbl}


% Für das Eurosymbol
\usepackage{eurosym}


\usepackage[bookmarks,bookmarksnumbered=TRUE]{hyperref} % Erstellt Hyperreferenzen im PDF-Dokument
\usepackage{tabularx}

%%%% Bild Numerirung
%\renewcommand{\thefigure}{\arabic{section}.\arabic{figure}}
%\usepackage{chngcntr}
%\counterwithin{figure}{section}
%%%% Tabellen Numerirung
%\renewcommand{\thetable}{\arabic{section}.\arabic{figure}}
%\usepackage{chngcntr}
%\counterwithin{table}{section}

%% Petrova
\usepackage{graphicx}
%neu von Timo
\usepackage{epstopdf}

%\textwidth 6in \textheight 9in \topmargin 0in \headsep 0in
%\oddsidemargin 0in \evensidemargin 0in

%\renewcommand{\theequation}{\thesection.\arabic{equation}}
%\newcommand{\address}[1]{\par\begin{center}{\sl #1} \end{center}}

%\setcounter{page}{1} \textwidth 6in \textheight 9in \topmargin 0in
%\headsep 0in \oddsidemargin 0in \evensidemargin 0in

%\newcommand{\res}[1]{\Big [\!\mbox{\begin{scriptsize}${#1}$\end{scriptsize}}\!\Big ]}

%\renewcommand{\theequation}{\thesection.\arabic{equation}}
%\newcommand{\address}[1]{\par\begin{center}{\sl #1} \end{center}}

%\newcommand {\R} {\textbf{\rm l\hskip-0.5mm R}}
%\newcommand {\N} {\textbf{\rm l\hskip-0.5mm N}}
%
\newcommand{\bA}{\mbox {\boldmath $A$}}
\newcommand{\bB}{\mbox {\boldmath $B$}}
\newcommand{\bF}{\mbox {\boldmath $F$}}
\newcommand{\bG}{\mbox {\boldmath $G$}}
\newcommand{\bH}{\mbox {\boldmath $H$}}
\newcommand{\bS}{\mbox {\boldmath $S$}}
\newcommand{\bV}{\mbox {\boldmath $V$}}
%
\newcommand{\bPhi}{\mbox {\boldmath $\Phi $}}
\newcommand{\bPsi}{\mbox {\boldmath $\Psi $}}
\newcommand{\bphi}{\mbox {\boldmath $\phi $}}
\newcommand{\bvarphi}{\mbox {\boldmath $\varphi $}}
\newcommand{\bsigma}{\mbox {\boldmath $\sigma $}}
\newcommand{\balpha}{\mbox {\boldmath $\alpha $}}
\newcommand{\bbeta}{\mbox {\boldmath $\beta $}}
\newcommand{\bxi}{\mbox {\boldmath $\xi $}}
\newcommand{\bpsi}{\mbox {\boldmath $\psi $}}
\newcommand{\bmu}{\mbox {\boldmath $\mu $}}
\newcommand{\bE}{\mbox {\boldmath $E $}}
\newcommand{\blambda}{\mbox {\boldmath $\lambda $}}
\newcommand{\bdelta}{\mbox {\boldmath $\delta $}}
\newcommand{\beeta}{\mbox {\boldmath $\eta $}}

\newcommand {\sbsigma} {{\small \bsigma}}
\newcommand {\sblambda } {{\small \blambda}}
\newcommand{\bx}{\mbox {\boldmath $x$}}
\newcommand{\by}{\mbox {\boldmath $y$}}
\newcommand{\bz}{\mbox {\boldmath $z$}}
\newcommand{\be}{\mbox {\boldmath $e$}}
\newcommand{\bu}{\mbox {\boldmath $u$}}
\newcommand{\br}{\mbox {\boldmath $r$}}
\newcommand{\bg}{\mbox {\boldmath $g$}}
\newcommand{\bff}{\mbox {\boldmath $f$}}
\newcommand{\bn}{\mbox {\boldmath $n$}}
\newcommand{\bv}{\mbox {\boldmath $v$}}
\newcommand{\bq}{\mbox {\boldmath $q$}}
\newcommand{\bt}{\mbox {\boldmath $t$}}
\newcommand{\bs}{\mbox {\boldmath $s$}}

\newcommand {\bb} {{\bf b}}
\newcommand {\bc} {{\bf c}}
\newcommand {\bd} {{\bf d}}
\newcommand {\bk} {{\bf k}}
\newcommand {\bw} {{\bf w}}
\newcommand {\bh} {{\bf h}}

